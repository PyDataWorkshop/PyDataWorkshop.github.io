### Data Structures
  \textbf{\textit{pandas}} introduces two new data structures to Python - \textbf{Series} and \textbf{DataFrame}, both of which are built on top of NumPy.
  
  
   \begin{verbatim}
   import pandas as pd
   import numpy as np
   import matplotlib.pyplot as plt
   pd.set_option('max_columns', 50)
   \end{verbatim}
  \end{framed}
  
  

  \frametitle{Series}
  
  Series is a one-dimensional labeled array capable of holding any data type (integers, strings, floating point numbers, Python objects, etc.). The axis labels are collectively referred to as the index. The basic method to create a Series is to call:
  
  
   \begin{verbatim}
   s = Series(data, index=index)
   \end{verbatim}
  \end{framed} 
  Here, data can be many different things:
  
   
    a Python \texttt{dict}
    an \texttt{ndarray}
    a scalar value (like 5)
   
  
  

   
   A Series is a one-dimensional object similar to an array, list, or column in a table. 
   It will assign a labeled index to each item in the Series.  By default, each item will receive an index label from 0 to N, where N is the length of the Series minus one.
   
  
  
  
   \begin{verbatim}
   # create a Series with an arbitrary list
   s = pd.Series([7, 'Heisenberg', 3.14, -1789710578,
        'Happy Eating!'])
   s
   \end{verbatim}
  \end{framed}

 
%         ======%
 [fragile]
\frametitle{Series}
\textbf{Output from Previous Slide}
\begin{framed}  
  \begin{verbatim}
  0                7
  1       Heisenberg
  2             3.14
  3      -1789710578
  4    Happy Eating!
  dtype: object
  \end{verbatim}
\end{framed}
  
 %         ======%
  [fragile]
  
  Alternatively, you can specify an index to use when creating the Series.
  
  
   \begin{verbatim}
   s = pd.Series([7, 'Heisenberg', 3.14, -1789710578, 
      'Happy Eating!'],
   index=['A', 'Z', 'C', 'Y', 'E'])
   s
   \end{verbatim}
  \end{framed}
  \begin{verbatim}
  A                7
  Z       Heisenberg
  C             3.14
  Y      -1789710578
  E    Happy Eating!
  dtype: object
  \end{verbatim}
  
 %         ======%
  [fragile]
\frametitle{Series}  
  The Series constructor can convert a dictonary as well, using the keys of the dictionary as its index.
  
  
   \begin{verbatim}
   d = {'Chicago': 1000, 'New York': 1300, 'Portland': 900, 'San Francisco': 1100,
   'Austin': 450, 'Boston': None}
   cities = pd.Series(d)
   cities
   Out[4]:
   Austin            450
   Boston            NaN
   Chicago          1000
   New York         1300
   Portland          900
   San Francisco    1100
   dtype: float64
   \end{verbatim}
  \end{framed}
  
 %         ======%
  [fragile]
\frametitle{Series}  
  You can use the index to select specific items from the Series ...
  
  
   \begin{verbatim}
   cities['Chicago']
   Out[5]:
   1000.0
   \end{verbatim}
  \end{framed}
  
 %         ======%
  [fragile]
\frametitle{Series}  
  
   \begin{verbatim}
   cities[['Chicago', 'Portland', 'San Francisco']]
   Out[6]:
   Chicago          1000
   Portland          900
   San Francisco    1100
   dtype: float64
   \end{verbatim}
  \end{framed}
  
 %         ======%
  [fragile]
\frametitle{Series}  
 You can use \textbf{\textit{boolean indexing}} for selection.
  
  
   \begin{verbatim}
   cities[cities < 1000]
   Out[7]:
   Austin      450
   Portland    900
   dtype: float64
   \end{verbatim}
  \end{framed}
  
 %         ======%
  [fragile]
  
  That last one might be a little strange, so let's make it more clear - \texttt{cities < 1000} returns a Series of \texttt{True/False} values, which we then pass to our Series cities, returning the corresponding \texttt{True} items.
  
  
 %         ======%
  [fragile]
  
  
   \begin{verbatim}
   less_than_1000 = cities < 1000
   print less_than_1000
   print '\n'
   print cities[less_than_1000]
   Austin            True
   Boston           False
   Chicago          False
   New York         False
   Portland          True
   San Francisco    False
   dtype: bool
   
   
   Austin      450
   Portland    900
   dtype: float64
  
   \end{verbatim}
  \end{framed}
  
 %         ======%
  [fragile]
  
  You can also change the values in a Series on the fly.
  
  
   \begin{verbatim}
   # changing based on the index

   print 'Old value:', cities['Chicago']

   cities['Chicago'] = 1400
   print 'New value:', cities['Chicago']

   Old value: 1000.0
   New value: 1400.0
   \end{verbatim}
  \end{framed}
 
%         ======%
 [fragile] 
Changing values using boolean logic
  
   \begin{verbatim}

   print cities[cities < 1000]
   print '\n'
   cities[cities < 1000] = 750
   
   print cities[cities < 1000]
   Austin      450
   Portland    900
   dtype: float64
   
   
   Austin      750
   Portland    750
   dtype: float64
   \end{verbatim}
  \end{framed}
  
 %         ======%
  [fragile]
  \frametitle{Working with Series}
  What if you aren't sure whether an item is in the Series? You can check using idiomatic Python.
  
  
   \begin{verbatim}
   print 'Seattle' in cities
   print 'San Francisco' in cities
   False
   True
   \end{verbatim}
  \end{framed}
  
 %         ======%
  [fragile]
  Mathematical operations can be done using scalars and functions.
  
  
   \begin{verbatim}
   # divide city values by 3
   cities / 3
   Out[12]:
   Austin           250.000000
   Boston                  NaN
   Chicago          466.666667
   New York         433.333333
   Portland         250.000000
   San Francisco    366.666667
   dtype: float64
  \end{verbatim}
 \end{framed}
 
   %         ======%
    [fragile]
   
   \begin{verbatim}
   # square city values
   np.square(cities)
   Out[13]:
   Austin            562500
   Boston               NaN
   Chicago          1960000
   New York         1690000
   Portland          562500
   San Francisco    1210000
   dtype: float64
   \end{verbatim}
  \end{framed}
  
 %         ======%
  [fragile]
  
  You can add two Series together, which returns a union of the two Series with the addition occurring on the shared index values. Values on either Series that did not have a shared index will produce a NULL/NaN (not a number).
  
  
   \begin{verbatim}
   print cities[['Chicago', 'New York', 'Portland']]
   print'\n'
   print cities[['Austin', 'New York']]
   print'\n'
   print cities[['Chicago', 'New York', 'Portland']] + cities[['Austin', 'New York']]
   \end{verbatim}
  \end{framed}
    

   %         ======%
    [fragile]
    \begin{verbatim}
  Chicago     1400
  New York    1300
  Portland     750
  dtype: float64
  
  
  Austin       750
  New York    1300
  dtype: float64
  
  
  Austin       NaN
  Chicago      NaN
  New York    2600
  Portland     NaN
  dtype: float64
  \end{verbatim}
  
 %         ======%
  [fragile]
 \frametitle{Working with Series}
 \textbf{NULL Checking}
  

\item Notice that because Austin, Chicago, and Portland were not found in both Series, they were returned with NULL/NaN values.

\item NULL checking can be performed with \texttt{isnull()} and \texttt{notnull()}.
    
  
 %         ======%
  [fragile]
  Return a boolean series indicating which values aren't NULL
  
  
   \begin{verbatim}
   cities.notnull()

   Austin            True
   Boston           False
   Chicago           True
   New York          True
   Portland          True
   San Francisco     True
   dtype: bool
   \end{verbatim}
  \end{framed}
 
%         ======%
%         ======%
   [fragile]
Using boolean logic to grab the NULL cities
  
  \begin{verbatim}
   print cities.isnull()
   print '\n'
   print cities[cities.isnull()]
   Austin           False
   Boston            True
   Chicago          False
   New York         False
   Portland         False
   San Francisco    False
   dtype: bool
      
   Boston   NaN
   dtype: float64
   \end{verbatim}
  \end{framed}
  
 %         ======%
\end{document}

%%---------------------------------------%
%\newpage
%\frametitle{DataFrame}
%
%% pandas - chapter 5 - DataFrame
%
%A DataFrame is a tablular data structure comprised of rows and columns, akin to a spreadsheet, database table, or R's data.frame object. You can also think of a DataFrame as a group of Series objects that share an index (the column names).
%
%%For the rest of the tutorial, we'll be primarily working with DataFrames.
%
%%---------------------------------------%
%\newpage
%\frametitle{Panel}
%
%
% 
%\texttt{Panel} is a somewhat less-used, but still important container for 3-dimensional data. 
%The term panel data is derived from econometrics and is partially responsible for the name pandas: pan(el)-da(ta)-s. 
%The names for the 3 axes are intended to give some semantic meaning to describing operations involving panel data and, 
%in particular, econometric analysis of panel data. However, for the strict purposes of slicing and dicing a 
%collection of DataFrame objects, you may find the axis names slightly arbitrary:
% 
% 
%\item items: axis 0, each item corresponds to a DataFrame contained inside
%\item major\_axis: axis 1, it is the index (rows) of each of the DataFrames
%\item minor\_axis: axis 2, it is the columns of each of the DataFrames
% 
%
%\newpage
%
%\end{document}
