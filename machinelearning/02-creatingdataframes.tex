
 

\section*{What is a Data Frame?}
 
\item A Dataframe is used for storing data tables.
\item It is a list of vectors of equal length.
\item The top line of the table, called the header, contains the column names.
\item Each horizontal line afterward denotes a data row, which begins with the name of the row, and then followed by the actual data.
\item Each data member of a row is called a \textbf{cell}.
\item A row is often called a \textbf{case}.
 

\newpage
\section*{pandas Workshop: Creating a DataFrame}
 
\item To create a DataFrame out of common Python data structures, we can pass a dictionary of lists to the \textit{DataFrame} constructor.

\item Using the columns parameter allows us to tell the constructor how we'd like the columns ordered. 
\item By default, the DataFrame constructor will order the columns alphabetically ( However this is not the case when reading from a file ).
 


\begin{framed}
\begin{verbatim}
data = {'year': [2013, 2014, 2015,2013, 2014, 2015, 2014, 2015, 
2013, 2014, 2015],
'club': ['United', 'United', 'United','Rovers', 'Rovers', 'Rovers', 
'Saints', 'Saints', 
'City', 'City', 'City'],
'wins': [5, 8, 6, 1,7, 7, 3, 5, 10, 6, 12] ,
'losses': [11, 8, 10, 15, 11, 10, 15, 11, 6, 10, 4] }

football = pd.DataFrame(data,
columns=['year', 'club', 'wins', 'losses'])

print football
\end{verbatim}
\end{framed}

\begin{verbatim}
   year     team  wins  losses
0  2010    Bears    11       5
1  2011    Bears     8       8
2  2012    Bears    10       6
3  2011  Packers    15       1
4  2012  Packers    11       5
5  2010    Lions     6      10
6  2011    Lions    10       6
7  2012    Lions     4      12

\end{verbatim}
%Much more often, you'll have a dataset you want to read into a DataFrame. Let's go through several common ways of doing so.

\end{document}
