\documentclass[MASTER.tex]{subfiles} 
\begin{document} 
 
%===========================================================%
 
 \frametitle{Versions of Python}
  
 Two Main Versions of Python
  
   Version 2.7
   Version 3
  
 
 
%===========================================================%
 
\frametitle{Python Coding Conventions}
 
There are a number of common practices which can be adopted to produce Python code which looks
more like code found in other modules:
 
  Use 4 spaces to indent blocks – avoid using tab, except when an editor automatically converts tabs
 to 4 spaces
  Avoid more than 4 levels of nesting, if possible
  Limit lines to 79 characters. The $\backslash$ symbol can be used to break long lines
 219
  Use two blank lines to separate functions, and one to separate logical sections in a function.
 
 
 
%===========================================================%
 
\frametitle{Python Coding Conventions}
 
  
  Use ASCII mode in text editors, not UTF-8
  One module per import line
  Avoid from module \texttt{import $\ast$} (for any module). Use either \texttt{from module import func1, func2} or
 \texttt{import module as shortname}.
  Follow the NumPy guidelines for documenting functions
 

% More suggestions can be found in PEP8.
 
%===========================================================%
\end{document}