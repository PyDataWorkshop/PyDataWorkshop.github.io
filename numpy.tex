\documentclass[MASTER.tex]{subfiles}

 
The <tt>{numpy} package}
 
 
*  The Python programming language was not initially designed for numerical computing, but attracted the attention of the scientific/engineering community early on.

*  NumPy is an extension to the Python programming language, adding support for large, multi-dimensional arrays and matrices, along with a large library of high-level mathematical functions to operate on these arrays. 
 
 
%==============%
 
The <tt>{numpy} package}
  
 
 
*  The ancestor of NumPy, Numeric, was originally created by Jim Hugunin with contributions from several other developers. 
*  In 2005, Travis Oliphant created NumPy by incorporating features of Numarray into Numeric with extensive modifications. 
 

 
%==============%
 
The <tt>{numpy} package}
  
 
 
*  NumPy is open source and has many contributors.
*  \textbf{Website} http://www.numpy.org/
 

 

%==============%
 
The <tt>{numpy} package}
  
\textbf{Useful Commands for simulation exercises}
 
*  <tt>{random.randint(a, b)} - Return a random integer N such that $a \leq N \leq b$.

*  <tt>{random.choice(seq)} - return a random element from the non-empty sequence <tt>{seq}. \\ If <tt>{seq} is empty, raises <tt>{IndexError}.

*  <tt>{random.sample(population, k)} - 
Return a k length list of unique elements chosen from the population sequence. Used for random \textit{sampling without replacement}.
 
 
%%===== %
% 
% \begin{figure}
%
%\includegraphics[width=1.16\linewidth]{numpyarraycreation}
%
%\end{figure}
%
%  
%==== %
% 
% \begin{figure}
%  
%  \includegraphics[width=1.15\linewidth]{numpybasicoperations}
%
% \end{figure}
% 
%  
%==== %
\end{document}